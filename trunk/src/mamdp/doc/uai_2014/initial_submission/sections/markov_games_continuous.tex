\section{Zero-sum Continuous Stochastic Games}
\label{sec:csg}

Continuous state stochastic games (CSGs) are formally defined by the tuple
$ \left\langle \vec{x}, A_{1}, \ldots, A_{n}, T, R_{1}, \ldots, R_{n}, h, \gamma  \right\rangle$.
In CSGs states are represented by vectors of continuous variables, $\vec{x} = \left(x_1, \ldots, x_m \right)$, 
where $x_i \in \mathbb{R}$. The other components of the tuple are as 
previously defined for discrete stochastic games in the preceding section.

The optimal solution to zero-sum CSGs can be calculated via the 
following recursive functions:
{\small 
\begin{eqnarray}
\label{eq:csgdiscqfunc}
  Q^{h}(\vec{x}, a_1, a_2) &=& R(\vec{x}, a_1, a_2) \quad + \nonumber \\
  && \gamma \cdot \int T(\vec{x}, a_1, a_2, \vec{x}') \cdot V^{h-1}(\vec{x}') d\vec{x}' 
\end{eqnarray}
}%

{\small 
\begin{equation}
\label{eq:csgvfunccompact}
  V^{h}(\vec{x}) = \max_{m_{a_{1}} \in \sigma(A_1)} \min_{a_2 \in A_2} \sum_{a_1 \in A_1} Q^{h}(\vec{x}, a_1, a_2) \cdot m_{a_{1}}
\end{equation}
}%

We can derive Equation \eqref{eq:csgdiscqfunc} from Equation \eqref{eq:dsgdiscqfunc}
by replacing the $s$, $s'$ and the $\sum$ operator with their continuous
state counterparts, $\vec{x}$, $\vec{x}'$ and $\int$, respectively.
%
%Here we note that Equation \eqref{eq:csgdiscqfunc} can be derived from Equation \eqref{eq:dsgdiscqfunc}
%by simply replacing $s$ and the $\Sigma$ operator in the former, with $\vec{x}$
%and the $\int$ operator in the latter.

\subsection{Solution Techniques}

Zero-sum CSGs can be solved using a technique analogous to that 
presented in preceding section. Namely, the value function in Equation
\eqref{eq:csgvfunccompact} can be reformulated as the following continuous 
optimisation problem:

{\small 
\begin{subequations}
\begin{align}
\text{maximise} &\   V^{h}(\vec{x}) \nonumber \\
\text{subject to} &\   V^{h}(\vec{x}) \leq \sum_{a_1 \in A_1} Q^{h}(\vec{x}, a_1, a_2) \cdot m_{a_{1}} \   \forall a_2 \in A_2 \label{eq:bilinearconstraint} \\
                          &\   m_{a_{1}} \geq 0 \qquad \forall a_{1} \in A_1 \nonumber\\
                          &\   \sum_{a_{1} \in A_1} m_{a_{1}} = 1 \nonumber        
\end{align}
\end{subequations}
}%

This optimisation problem cannot be easily solved using existing techniques
due to two factors: (1) there are infinitely many states in $\vec{x}$; and
(2) constraint \eqref{eq:bilinearconstraint} is nonlinear in $\vec{x}$ and 
$m_{a_{1}}$ for general representations of {\small $Q^{h}(\vec{x}, a_1, a_2)$}. 
To further illustration the second limitation consider 
$Q^{h}(\vec{x}, a_1, a_2)$ in the form of a linear function in $x$ for some
$a_1$ and $a_2$:

{\small 
\begin{equation}
Q^{h}(\vec{x}, a_1, a_2) = \sum_{j} c_j \cdot x_j \label{eq:linqfunc}
\end{equation}
}%

Substituting Equation \eqref{eq:linqfunc} into constraint \eqref{eq:bilinearconstraint}
yields:

{\small 
\begin{equation}
V^{h}(\vec{x}) \leq \sum_{a_1 \in A_1} m_{a_{1}} \sum_{j} c_j \cdot x_j \qquad \forall a_2 \in A_2. \label{eq:linqfuncresult}
\end{equation}
}%

It is clear from Equation \eqref{eq:linqfuncresult} that a linear representation
of $Q^{h}(\vec{x}, a_1, a_2)$ leads to a nonlinear constraint
where $m_{a_{1}}$ must be optimal with respect to the free variable
$\vec{x}$ since we need to solve for all states $\vec{x}$. This results in 
a parameterised nonlinear program, whose optimal solutions are known to be
NP-hard \cite{Bennett_COA_1993,Petrik_JoMLR_2011}.

At this point we present the first key insight of this paper: we
can transform constraint \eqref{eq:bilinearconstraint} from a parameterised 
nonlinear constraint to a piecewise linear constraint by imposing the 
following restrictions: (1) restricting the reward function, {\small $R(\vec{x}, a_1, a_2)$}, 
to piecewise constant functions; and (2) restricting the transition function, 
{\small $T(\vec{x}, a_1, a_2, \vec{x}')$}, to piecewise linear functions.

In the next section we show that zero-sum CSGs, with the aforementioned
restrictions, can be solved optimally for arbitrary horizons using 
symbolic dynamic programming \cite{Zamani_AAAI_2012}.