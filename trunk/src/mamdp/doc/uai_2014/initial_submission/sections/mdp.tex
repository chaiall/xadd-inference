\section{Markov Decision Processes}
\label{sec:mdp}

A Markov Decision Process (MDP) \cite{Howard_1960} is defined by the tuple
$ \left\langle S, A, T, R, h, \gamma \right\rangle$. $S$ and $A$ 
specify a finite set of states and actions, respectively.
$T$ is a transition function $T : S \times A \rightarrow S$ which 
defines the effect of an action on the state. $R$ is the
reward function $R : S \times A \rightarrow \mathbb{R}$ which 
encodes the preferences of the agent. The horizon $h$ represents the 
number of decision steps until termination and the discount factor $\gamma \in [0, 1)$ 
is used to discount future rewards. In general, an agent's objective is 
to find a policy, $\pi : S \rightarrow A$, which maximises the expected 
sum of discounted rewards over horizon $h$.

Value iteration (VI) \cite{Bellman_1957} is a general dynamic programming 
algorithm used to solve MDPs. VI is based on the set of Bellman equations,
which mathematically express the optimal solution of an MDP. They 
provide a recursive expansion to compute: (1) $V^{*}(s)$, the expected value of following
the optimal policy in state $s$; and (2) $Q^{*}(s, a)$, which is the expected
quality of taking $a$ in state $s$, then following the optimal policy. The
key idea of the algorithm is to successively approximate $V^{*}(s)$ and $Q^{*}(s, a)$
by $V^{h}(s)$ and $Q^{h}(s, a)$, respectively, at each horizon $h$. These 
two functions satisfy the following recursive relationship:

{\small 
\abovedisplayskip=0pt
\belowdisplayskip=0pt
\begin{align}
  Q^{h}(s, a) &= R(s, a) + \gamma \cdot \sum_{s' \in S} T(s, a, s') \cdot V^{h-1}(s') \label{eq:qfunc}\\
  V^{h}(s) &= \max_{a \in A} \left\{ Q^{h}(s, a) \right\} \label{eq:vfunc}
\end{align}
}%

The algorithm is executed by first initialising $V^{0}$  to zero or the terminal reward. 
Then for each $h$, $V^{h}(s)$ is calculated from $V^{h-1}(s)$ via
Equations \eqref{eq:qfunc} and \eqref{eq:vfunc}, until the intended 
$h$-stage-to-go value function is computed. Value iteration converges 
linearly in the number of iterations to the true values of $Q^{*}(s, a)$ and $V^{*}(s)$ 
\cite{Bertsekas_1987}.

MDPs can be used to model multiagent systems under the assumption 
that other agents are part of the environment and have fixed behaviour. 
As a result, they ignore the difference between responsive agents and 
a passive environment \cite{Hu_ICML_1998}. In the next section we 
present a game theoretic framework which generalises MDPs to 
situations with two or more agents.