\section{Symbolic Dynamic Programming}
\label{sec:sdp}

Symbolic dynamic programming (SDP) \cite{Boutilier_IJCAI_2001} is 
the process of performing dynamic programming via symbolic 
manipulation. In the following sections we present a brief overview
of SDP operations and also show how SDP can be used to solve
zero-sum CSGs.

\subsection{Case Representation}

SDP assumes that all functions can be represented in case statement form \cite{Boutilier_IJCAI_2001} as follows:

{\small 
\abovedisplayskip=0pt
\belowdisplayskip=0pt
\begin{align*}
  f = 
    \begin{cases}
      \phi_1: & f_1 \\ 
      \vdots & \vdots\\ 
      \phi_k: & f_k \\ 
    \end{cases}
\end{align*}
}%

Here, the $\phi_i$ are logical formulae defined over the state $\vec{x}$ 
that can consist of arbitrary logical combinations of boolean variables and 
linear inequalities $\left( \geq, >, <, \leq \right)$ over continuous variables. We assume that the set of
conditions $\left\lbrace \phi_1, \ldots, \phi_k \right\rbrace$ disjointly
and exhaustively partition $\vec{x}$ such that $f$ is well-defined for all $\vec{x}$.
In general, the $f_i$ may be polynomials of $\vec{x}$ with non-negative exponents.
However, in this paper we restrict the $f_i$ to be either constant or linear functions 
of the state variable $\vec{x}$. Henceforth, we refer to functions with 
linear $\phi_i$ and piecewise constant $f_i$ as linear piecewise constant (LPWC) 
and functions with linear $\phi_i$ and piecewise linear $f_i$ 
as linear piecewise linear (LPWL) functions.

\subsection{Case Operations}

Operations on case statements may be either unary or binary. In this section
we present a brief overview of the SDP operations needed to calculate closed form
solutions to zero-sum CSGs. All of the operations presented here are closed form
for LPWC and LPWL functions. We refer the reader to \cite{Sanner_UAI_2011,Zamani_AAAI_2012} 
for more thorough expositions of SDP and its operations.

Unary operations on a aingle case statement \emph{f}, such as scalar 
multiplication $c \cdot f$ where $ c \in \mathbb{R} $, are applied to each $f_i$ ($1 \leq i \leq k$). 

Binary operations such as addition, subtraction and multiplication are executed 
in two stages. Firstly, the cross-product of the logical partitions of each case statement 
is taken, producing paired partitions. Finally, the binary operation 
is applied to the resulting paired partitions. The ``cross-sum'' $\oplus$
operation can be performed on two cases in the following manner:

{\small 
\abovedisplayskip=0pt
\belowdisplayskip=0pt
\begin{center}
  \begin{tabular}{r c c c l}
    $\begin{cases}
        \phi_1: \hspace{-1mm} & \hspace{-1mm} f_1  \\ 
        \phi_2: \hspace{-1mm} & \hspace{-1mm} f_2  \\ 
    \end{cases}$
  $\oplus$
  &
  \hspace{-4mm}
    $\begin{cases}
        \psi_1: \hspace{-1mm} & \hspace{-1mm} g_1  \\ 
        \psi_2: \hspace{-1mm} & \hspace{-1mm} g_2  \\ 
    \end{cases}$
  &
  \hspace{-4mm} 
  $ = $
  &
  \hspace{-4mm}
    $\begin{cases}
      \phi_1 \wedge \psi_1: & f_1 + g_1 \\
      \phi_1 \wedge \psi_2: & f_1 + g_2 \\
      \phi_2 \wedge \psi_1: & f_2 + g_1 \\
      \phi_2 \wedge \psi_2: & f_2 + g_2  \\
    \end{cases}$
  \end{tabular}
\end{center}
}%

``cross-subtraction''  $\ominus$ and ``cross-multiplication'' $\otimes$
are defined in a similar manner but with the addition operator replaced
by the subtraction and multiplication operators, respectively.
Some partitions resulting from case operators may be inconsistent and 
are thus removed. 

Minimisation over cases, known as $\casemin$, is defined as:

{\footnotesize 
\abovedisplayskip=0pt
\belowdisplayskip=0pt
\begin{center}
  \begin{tabular}{r c c c l}
  \hspace{-7mm} 
  
  $\casemin \Bigg(
    \begin{cases}
        \phi_1: \hspace{-1mm} & \hspace{-1mm} f_1 \\ 
        \phi_2: \hspace{-1mm} & \hspace{-1mm} f_2 \\ 
    \end{cases}$
  $,$
  &
  \hspace{-4mm}
    $\begin{cases}
        \psi_1: \hspace{-1mm} & \hspace{-1mm} g_1 \\ 
        \psi_2: \hspace{-1mm} & \hspace{-1mm} g_2 \\ 
    \end{cases} \Bigg)$
  &
  \hspace{-4mm} 
  $ = $
  &
  \hspace{-4mm}
    $\begin{cases}
      \phi_1 \wedge \psi_1 \wedge f_1 < g_1    : & \hspace{-2mm} f_1 \\ 
      \phi_1 \wedge \psi_1 \wedge f_1 \geq g_1 : & \hspace{-2mm} g_1 \\ 
      \phi_1 \wedge \psi_2 \wedge f_1 < g_2    : & \hspace{-2mm} f_1 \\ 
      \phi_1 \wedge \psi_2 \wedge f_1 \geq g_2 : & \hspace{-2mm} g_2 \\ 
      \vdots & \vdots
    \end{cases}$
  \end{tabular}
\end{center}
}%

$\casemin$ preserves the linearity of the constraints and the constant 
or linear nature of the $f_i$ and $g_i$. If the $f_i$ or 
$g_i$ are quadratic then the expressions $f_i > g_i$ or 
$f_i \leq g_i$ will be at most univariate quadratic and any such 
constraint can be linearised into a combination of at most two linear 
inequalities by completing the square. 

Substitution into case statements is performed via a set $\theta$ of variables and their
substitutions e.g. {\small $\theta = \left\{ x'_1/(x_1 + x_2), x'_2/x^2_{1} \text{exp}(x_2) \right\}$},
where the LHS of the / represents the substitution variable and the 
RHS of the / represents the expression that should be substituted in its place.
$\theta$ can be applied to both non-case functions $f_i$ and case 
partitions $\phi_i$ as $f_i\theta$ and $\phi_i\theta$, respectively.
Substitution into case statements can be written as:

{\small 
\abovedisplayskip=0pt
\belowdisplayskip=0pt
\begin{align*}
  f\theta = 
    \begin{cases}
      \phi_1\theta: & f_1\theta \\ 
      \vdots & \vdots\\ 
      \phi_k\theta: & f_k\theta \\ 
    \end{cases}
\end{align*}
}%

Substitution is used when calculating integrals with respect to  deterministic $\delta$
transitions \cite{Sanner_UAI_2011}.

A case statement can be maximised with respect to a continuous parameter $y$
as {\small $ f_1(\vec{x}, y) = \contmax_{y}f_2(\vec{x}, y) $}. The continuous maximisation
operation is a complex case operation whose explanation is beyond the scope of 
this paper. We refer the reader to \cite{Zamani_AAAI_2012} for further details.

Case statements and their operations are implemented using Extended 
Algebraic Decision Diagrams (XADDs) \cite{Sanner_UAI_2011}.
XADDs provide a compact data structure with which to maintain
compact forms of {\small $Q^{h}(\vec{x}, a_1, a_2)$} and {\small $V^{h}(\vec{x})$}. 

\subsection{SDP for Zero-sum Continuous Stochastic Games}

In this section we will show that a subclass of zero-sum continuous stochastic
games with (a) piecewise constant rewards; and (b) piecewise linear transition
functions can be solved exactly and in closed-form by using SDP.

To calculate the exact solution to zero-sum CSGs we begin by replacing all functions
and operations in Equations \eqref{eq:csgdiscqfunc} and \eqref{eq:csgvfunccompact}
by their case statement equivalents. That is, we exchange operations such as 
$+$, $\times$ and min, by their symbolic equivalents, $\oplus$, $\otimes$ and $\casemin$,
respectively, and express {\small $R(\vec{x}, a_1, a_2)$}, 
{\small $T(\vec{x}, a_1, a_2, \vec{x}')$}, {\small $V^0(\vec{x})$} as case statements. $m_{a_{1}}$
is also encoded as a trivial case statement representing an uninstantiated symbolic variable:

{\small 
\abovedisplayskip=0pt
\belowdisplayskip=0pt
\begin{align*}
  m_{a_{1}} = 
    \begin{cases}
      \top: & m_{a_{1}} \\ 
    \end{cases}
\end{align*}
}%

The optimal solution to zero-sum CSGs can now be described by the following
recursive SDP equations:

{\small 
\abovedisplayskip=0pt
\belowdisplayskip=0pt
\begin{align}
  Q^{h}(\vec{x}, a_1, a_2) &= R(\vec{x}, a_1, a_2) \quad \oplus \nonumber \\
  & \gamma \otimes \int T(\vec{x}, a_1, a_2, \vec{x}') \otimes V^{h-1}(\vec{x}')\ d\vec{x}' \label{eq:sdpdiscqfunc} \\
  \tilde{Q}^{h}(\vec{x}, a_2) &= \sum_{a_1 \in A_1} Q^{h}(\vec{x}, a_1, a_2) \otimes m_{a_{1}} \label{eq:sdpdiscqfunc2}
\end{align}
}%

{\small 
\abovedisplayskip=0pt
\belowdisplayskip=0pt
\begin{align}
\label{eq:sdpvfunccompact}
& V^{h}(\vec{x}) = \nonumber \\
& \max_{m} \casemin \left( \casemin_{a_2 \in A_2} \left( \tilde{Q}^{h}(\vec{x}, a_2) \right), \mathbb{I} \right)
\end{align}
}%

Equation \eqref{eq:sdpdiscqfunc2} calculates a symbolic $Q$ function weighted 
by the variable $m_{a_{1}}$ for each $a_1$. In Equation \eqref{eq:sdpvfunccompact} the inner $\casemin$ operation 
is calculated with respect to {\small $\tilde{Q}^{h}(\vec{x}, a_2)$} instantiated 
with a particular $a_2$. The ``indicator'' function  $\mathbb{I}$ serves as the summation constraint 
{\small $\sum_{a_{1} \in A_1} m_{a_{1}} = 1$} and ensures that the subsequent $\contmax$ operation returns 
legal values for the $m_{a_{1}}$. The indicator is defined as follows:

{\small
\abovedisplayskip=0pt
\belowdisplayskip=0pt
\begin{align*}
  & \mathbb{I} = \\
  & {\footnotesize
    \begin{cases}
      \forall a_{1} \in A_1 \hspace{6pt}\left[(m_{a_{1}} \geq 0) \wedge (m_{a_{1}} \leq 1 ) \wedge (\sum m_{a_{1}} = 1) \right]: +\infty \\ 
      \forall a_{1} \in A_1 \neg \left[(m_{a_{1}} \geq 0) \wedge (m_{a_{1}} \leq 1 ) \wedge (\sum m_{a_{1}} = 1) \right]: -\infty \\ 
    \end{cases} 
   }
\end{align*}
}%

The function $\mathbb{I}$ returns $+\infty$ when the conjunction of all constraints on each $m_{a_1}$ are satisfied
and $-\infty$, otherwise.

In Algorithm~\ref{alg:csgvi} we present \texttt{CSG-VI}, a methodology to calculate the optimal $h$-stage-to-go value function
through Equations ~\eqref{eq:sdpdiscqfunc} to ~\eqref{eq:sdpvfunccompact}. In the algorithm we notationally specify 
the arguments of the $Q^h$ and $V^h$ functions when they are 
instantiated by an operation. For the base case of $h = 0$, we set {\small $V^0(\vec{x})$}, expressed in 
case statement form, to zero or to the terminal reward. For all $h > 0$
we apply the previously defined SDP operations in the following steps:
\begin{enumerate}
  \item Prime the Value Function. 
            In line \ref{alg:prime} we set up a 
            substitution {\small $\theta = \left\{ x_1/x'_1, \ldots, x_m/x'_m \right\}$}, 
            and obtain {\small $V^{h'} = V^{h}\theta$}, the ``next state''.
  \item Discount and add Rewards. 
            We assume that the reward function contains primed variables and incorporate it in line \ref{alg:dr1}.
  \item Continuous Regression. For the continuous state variables in $\vec{x}$
            lines \ref{alg:cmi1} -- \ref{alg:cmi2} follow the rules of integration w.r.t.~a $\delta$ function 
            \cite{Sanner_UAI_2011}. This yields the following symbolic substitution: 
            {\small $\int f(x'_j) \otimes \delta\left[ x'_j - g(\vec{z})\right] dx'_j = f(x'_j)\left\{ x'_j/g(\vec{z})\right\}$},
            where $g(\vec{z})$ is a case statement and $\vec{z}$ does not contain $x'_j$.
            The latter operation indicates that any occurrence of $x'_j$ in $f(x'_j)$ is symbolically substituted
            with the case statement $g(\vec{z})$.
  \item Incorporate Agent 1's strategy. 
            At this point we have calculated {\small $Q^h(\vec{x}, a_1, a_2)$}, as shown in
            Equation \eqref{eq:sdpdiscqfunc}. In lines \ref{alg:dm1} - \ref{alg:dm2} 
            we weight the strategy for a specific $a_1$ by $m_{a_{1}}$.
            We note that $m_{a_{1}}$ is a case statement representing
            an uninstantiated symbolic variable.
  \item Case Minimisation. 
            In lines \ref{alg:cm1} -- \ref{alg:cm2} we compute the best case for 
            $a_2$ as a function of all other variables, as shown in Equation~\eqref{eq:sdpdiscqfunc2}. 
  \item Incorporate Constraints. 
            In line \ref{alg:constraint} we incorporate constraints on the $m_{a_{1}}$ variable: 
            {\small $\sum_{a_{1} \in A_1} m_{a_{1}} = 1$} and 
            {\small $m_{a_{1}} \geq 0 \wedge m_{a_{1}} \leq 1 \hspace{10pt} \forall a_{1} \in A_1 $}.
            The $\casemin$ operator ensures that all case partitions which involve illegal values of 
            $m_{a_{1}}$ are mapped to $-\infty$.
  \item Maximisation. 
            In lines \ref{alg:max1} -- \ref{alg:max2} we compute the best response strategy for every state. We note that this can only be done
            via symbolic methods since there are infinitely many states. At this point we have calculated $V^{h}(\vec{x})$ as shown in 
            Equation \eqref{eq:sdpvfunccompact}.
\end{enumerate}

%%%%%%%%%%%%%%%%%%%%%%%%%%%%%%%%%
% Algorithm: CSG-VI
%%%%%%%%%%%%%%%%%%%%%%%%%%%%%%%%%
\incmargin{1.5em}
\linesnumbered
\begin{algorithm}[ht!]
  \vspace{-.5mm}
  \dontprintsemicolon
  \SetKwFunction{remapWithPrimes}{Prime}
  
  \Begin{
  
    $V^0:=0, h:=0$\;

    \While{$h < H$}{
      $h := h + 1$\;
      \emph{// Prime the value function}\;
      $Q^h := \remapWithPrimes{$V^{h-1}$} \,$ \; \label{alg:prime}

      \emph{// Discount and add Rewards }\;
        $Q^h := R(\vec{x}, a_1, a_2, \vec{x}') \oplus (\gamma \otimes Q^h)$ \; \label{alg:dr1} 
      
      \emph{// Continuous Regression}\;
      \For {all $x'_j \in Q^h$}{ \label{alg:cmi1}
        $Q^h := \int Q^h \otimes T(x_j'|a_1, a_2, \vec{x}) \hspace{3pt} d_{x'_j}$\; \label{alg:cmi2}
      }

      \emph{// Incorporate Agent 1's strategy }\;
      \For {all $a_1 \in A_1$}{ \label{alg:dm1}
        $Q^h := Q^h \oplus \left( Q^h\left(a_1\right) \otimes \{ \top : m_{a_{1}} \right) $ \; \label{alg:dm2}
      }

      \emph{// Case Minimisation}\;
      \For {all $a_2 \in A_2$}{ \label{alg:cm1}
        $Q^h := \casemin(Q^h, Q^h\left(a_2\right)) $ \; \label{alg:cm2}
      }

      \emph{// Incorporate constraints }\;
      $Q^h := \casemin(Q^{h}, \mathbb{I}) $\; \label{alg:constraint}

      \emph{// Maximisation}\;
      $V^h = Q^h$\;
      \For {all $a_1 \in A_1$}{ \label{alg:max1}
      $V^h := \contmax_{m_{a_{1}}} V^h\left(m_{a_{1}} \right)$ \; \label{alg:max2}
      }

       \emph{// Terminate if early convergence}\;
      \If{$V^h = V^{h-1}$} {
        break 
        $\,$
      }
    }    
    \Return{$(V^h)$} \;
  }
  \caption{
    \footnotesize \texttt{CSG-VI}(CSG, $H$, $\mathbb{I}$) $\longrightarrow$ $(V^h)$ 
    \label{alg:csgvi}
  }
  \vspace{-1mm}
\end{algorithm}
\decmargin{.5em}
%%%%%%%%%%%%%%%%%%%%%%%%%%%%%%%%%
% Algorithm: CSG-VI
%%%%%%%%%%%%%%%%%%%%%%%%%%%%%%%%%

It can be proved that all of the SDP operations used in Algorithm 1 are
closed form for LPWC or LPWL functions \cite{Sanner_UAI_2011,Zamani_AAAI_2012}. 
Given a LPWC  {\small $V^0(\vec{x})$} and that SDP operations are closed
form, the resulting {\small $V^{h+1}(\vec{x})$} is also LPWC.
Therefore, by induction {\small $V^{h+1}(\vec{x})$} is LPWC for all horizons $h$.

In the next section we demonstrate how SDP can be used to compute
exact solutions to a variety of zero-sum continuous stochastic games.
