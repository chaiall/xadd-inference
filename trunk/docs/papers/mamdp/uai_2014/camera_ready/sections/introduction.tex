\section{Introduction}

Modelling competitive sequential interactions between agents has
important applications within economic and financial decision-making.
Stochastic games \cite{Shapley_PotNAoS_1953} provide a convenient
framework to model sequential interactions between non-cooperative
agents. In zero-sum stochastic games, participating agents have
diametrically opposing goals. A reinforcement learning solution to
discrete state zero-sum stochastic games was presented by Littman
\cite{Littman_ICML_1994}.  Closed-form solutions for the continuous
state case remain unknown, despite the general importance of this
formalism --- zero-sum continuous state stochastic games provide a
convenient framework with which to model robust sequential
optimisation in adversarial settings including domains such as option valuation on derivative
markets.

The difficulty of solving zero-sum continuous state stochastic games
originates from the need to calculate a Nash equilibrium for every state,
of which there are infinitely many. In this paper we make the following 
key contributions:
\begin{itemize}
  \item We characterise a subclass of zero-sum continuous state stochastic
    games with restricted reward and transition functions that can be
    solved exactly via parameterised linear optimisation.
  \item We provide an algorithm that solves this subclass of
    stochastic games exactly and optimally using Symbolic Dynamic
    Programming (SDP)~\cite{Boutilier_IJCAI_2001,Sanner_UAI_2011,Zamani_AAAI_2012}
     for arbitrary horizons.
\end{itemize}

This paper is organised as follows: In Section ~\ref{sec:mdp} we describe Markov 
Decision Processes (MDPs)~\cite{Howard_1960} and value iteration~\cite{Bellman_1957}, 
a widely used dynamic programming method for solving MDPs. In Sections~\ref{sec:dsg} and
~\ref{sec:csg}, we present zero-sum stochastic games with discrete and continuous states, respectively,
as game-theoretic generalisations of the MDP framework. Following this, in Section ~\ref{sec:sdp}, we 
introduce SDP, and show how it can be used to calculate the first known exact solution 
to a particular subclass of zero-sum continuous state stochastic games. In Section~\ref{sec:results} 
we calculate exact solutions to three empirical domains: a continuous state generalisation of matching pennies, 
binary option valuation and robust energy production. In Section~\ref{sec:relatedwork}, we
survey the related literature. We conclude in Section~\ref{sec:conclusion} and identify
interesting directions for future research.