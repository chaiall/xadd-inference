\section{Continuous Stochastic Games}
\label{sec:csg}

Continuous state stochastic games (CSGs) are formally defined by the tuple \\
$ \left\langle \vec{x}, A_{1 \ldots n}, T, R_{1 \ldots n} \right\rangle $.
States are represented by a vectors of continuous variables, $\vec{x} = \left(x_1, \ldots, x_m \right)$, 
where $x_i \in \mathbb{R}$. The other components of the tuple are
as previously defined for DSGs in Section \ref{sec:dsg}.

Zero-sum continuous stochastic games can be defined by the following
recursive functions
\begin{equation}
\label{eq:csgvfunccompact}
  V^{h}(\vec{x}) = \max_{\pi_{a_1} \in \sigma(A_1)} \min_{a_2 \in A_2} \sum_{a_1 \in A_1} Q^{h}(\vec{x}, a_1, a_2) \cdot \pi_{a_1}.
\end{equation}

\begin{equation}
\label{eq:csgdiscqfunc}
  Q^{h}(\vec{x}, a_1, a_2) = R(\vec{x}, a_1, a_2) + \gamma \int T(\vec{x}, a_1, a_2, \vec{x}') V^{h-1}(\vec{x}') d\vec{x}' .
\end{equation}

%Discrete state stochastic games can be generalised to the continuous case by
%representing state $s$ by a vector of continuous state variables 
%$\vec{x} = \left(x_1, \ldots, x_m \right)$.

%Incorporating $\vec{x}$ into the value function in Equation \ref{eq:discvfunc} gives
%
%\begin{equation}
%\label{eq:contvfunc}
  %V^{h-1}(\vec{x}) = \max_{\pi_{a_1} \in \sigma(A_1)} \left[ \min_{a_2 \in A_2} \int \sum_{a_1 \in A_1} Q(\vec{x}, a_1, a_2) \cdot \pi_{a_1} d\vec{x} \right]. 
%\end{equation}
%
%We note that the function contained within $\left[ \cdot \right]$ is essentially
%$V^{h}(\vec{x})$, the value function in the future, that is,
%
%\begin{equation}
%V^{h}(\vec{x}) = \min_{a_2 \in A_2} \sum_{a_1 \in A_1} Q(\vec{x}, a_1, a_2) \cdot \pi_{a_1}. 
%\end{equation}
%
%The continuous state equivalent of Equation \ref{eq:discqfunc} is given
%by
%
%\begin{equation}
%\label{eq:contqfunc}
  %Q(\vec{x}, a_1, a_2) = R(\vec{x}, a_1, a_2) + \gamma \sum_{s' \in S} T(\vec{x}, a_1, a_2, s') V(\vec{x}'). \\
%\end{equation}

\subsection{Solution Techniques}

Equation \ref{eq:csgvfunccompact} can be solved using a technique 
analogous to that presented in Section \ref{subsec:dsgsolution}. 
Namely, the CSG formulation of the value function can be re-written 
as the following optimisation problem:

\begin{subequations}
\begin{align}
\text{maximise} & \qquad V(\vec{x}) \\
\text{subject to} & \qquad \pi_{a_1} \geq 0 \qquad \forall a_1 \in A_1 \\
                        & \qquad \sum_{a_1 \in A_1} \pi_{a_1} = 1 \\
                        & \qquad V(\vec{x}) \leq \sum_{a_1 \in A_1} Q(\vec{x}, a_1, a_2) \cdot \pi_{a_1} \qquad \forall a_2 \in A_2 \label{eq:bilinearconstraint}
\end{align}
\end{subequations}

This optimisation problem cannot be easily solved using existing techniques
due to two factors: there are infinitely many states in $\vec{x}$ and
constraint \ref{eq:bilinearconstraint} is bilinear in $\vec{x}$ and 
$\pi_{a_1}$. Solving bilinear programs optimally is known to be
NP-hard \cite{Bennett_COA_1993, Petrik_JoMLR_2011}.

We note that by restricting the reward function $ R(\vec{x}, a_1, a_2) $ 
to be piecewise constant, the previously bilinear constraint \ref{eq:bilinearconstraint}
is made piecewise linear. This resulting subclass of zero-sum continuous stochastic
games can then be solved optimally a given horizon $h$ using symbolic 
dynamic programming techniques \cite{Zamani_AAAI_2012}.

%-  So your contribution is simply in identifying what form of Q(x) gives a solution you know how to compute 
% proving that Q(x) will retain this form for an arbitrary horizon... 
% - so the solution stays closed-form and computable for any horizon (otherwise it is not a computable solution).

In the next section we present an overview of symbolic dynamic 
programming, its implementation, and show how it can be used to 
calculate exact solutions to this subclass of continuous zero-sum 
stochastic games.