We first formally introduce the framework of Hybrid (discrete and
continuous) Markov decision processes with non-deterministic
continuous noise (ND-HMDPs) by extending the HMDP framework 
of~\cite{sdp_aaai}. The optimal solution for this model is then defined
via robust dynamic programming.

\subsection{Factored Representation}

An HMDP is modelled using state variables $(\vec{b},\vec{x}) = (
b_1,\ldots,b_a,x_{1},\ldots,x_c )$ where each $b_i \in \{ 0,1 \}$
($1 \leq i \leq a$) represents a discrete boolean variable and
each $x_j \in \mathbb{R}$ ($1 \leq j \leq c$) is continuous.  To model
continuous uncertainty in ND-HMDPs we additionally define intermediate
noise variables $\vec{n} = n_1, \ldots , n_e$ where each
$n_l \in \mathbb{R}$ ($1 \leq l \leq e$).  Both discrete and
continuous actions are represented in the set $A
= \{a_1(\vec{y}_1), \ldots, a_p(\vec{y}_p) \}$ where each action
$a(\vec{y}) \in A$ references a (possibly empty) vector of continuous
parameters $\vec{y} \in \mathbb{R}^{|\vec{y}|}$; we say an action is
discrete if it has no continuous parameters ($|\vec{y}| = 0$), otherwise
it is continuous.

Given a current state $(\vec{b},\vec{x})$ and next state
$(\vec{b}',\vec{x}')$ and an executed action $a(\vec{y})$ 
at the current state, a real-valued reward function $R(\vec{b},\vec{x},\vec{b}',\vec{x}',a,\vec{y})$
specifies the immediate reward obtained at the current state. The probability of the
next state $(\vec{b}',\vec{x}')$ is defined by a joint state
transition model
$P(\vec{b}',\vec{x}'| \vec{b},\vec{x},a,\vec{y},\vec{n})$ which
depends on the current state, action and noise.
In a factored setting, we do not typically represent the transition
distribution jointly but rather we factorize it into a
dynamic Bayes net (DBN)~\cite{dbn} as follows:
{\footnotesize
\begin{align}
P&(\vec{b}',\vec{x}'|\vec{b},\vec{x}, a,\vec{y},\vec{n}) = \nonumber  \\
& \prod_{i=1}^a P(b_i'|\vec{b},\vec{x},\vec{b}',\vec{x}',a,\vec{y},\vec{n}) 
  \prod_{j=1}^c P(x_j'|\vec{b},\vec{x},\vec{b}',\vec{x}',a,\vec{y},\vec{n})
\end{align}
}
\emph{Here we allow synchronic arcs under the condition that the DBN forms
a proper directed acyclic graph (DAG)}.
For binary variables $b_i$ ($1 \leq i \leq a$),
$P(b_i'|\vec{b},\vec{x},\vec{b}',\vec{x}',\vec{x},a,\vec{y},\vec{n})$ are defined as
general conditional probability functions (CPFs), which are not necessarily tabular
since they may condition on inequalities over continuous variables.  For
continuous variables $x_j$ ($1 \leq j \leq c$), the CPFs
$P(x_j'|\vec{b},\vec{x},\vec{b}',\vec{x}',a,\vec{y},\vec{n})$ are represented
with \emph{piecewise linear equations} (PLEs) that may have piecewise 
conditions which are arbitrary logical combinations of
$\vec{b}$, $\vec{b}'$ and linear inequalities over $\vec{x}$, $\vec{x}'$,
and $\vec{n}$.  Examples of PLEs will follow shortly.

In general, we assume that for each intermediate continuous noise variable 
$n_l$ ($1 \leq l \leq e$)
a non-deterministic noise interval constraint function $N(n_l| \vec{b},\vec{x},a,\vec{y})$ 
has been defined 
that represents a range covering $\alpha$ of the probability mass for $n_l$ and evaluates to 
$-\infty$ for legal values of $n_l$ and $+\infty$ otherwise.  The reason for
the $\pm \infty$ evaluation is simple: 
in a robust solution to HMDPs with non-deterministic noise constraints, Nature will
attempt to adversarially minimize the reward the agent can achieve and hence we let
$N(n_l| \vec{b},\vec{x},a,\vec{y})$ take the value $+\infty$ for illegal values
of $n_l$ to ensure Nature will never choose illegal assignments of $n_l$ when minimizing.

%for an arbitrary constraint set $C(\vec{b},\vec{x},a,\vec{y})$ 
%and 0-1 indicator function $\I$, 
%this can be defined mathematically as:
%\begin{equation}
%N(n_l| \vec{b},\vec{x},a,\vec{y}) = 
%\begin{cases}
%\max_C \I[C\vec{b},\vec{x},a,\vec{y})] \int_{n_l} P(n_l| \vec{b},\vec{x},a,\vec{y}) dn_l > \alpha: & -\infty \\
%\text{otherwise} : & +\infty 
%\end{cases} \label{eq:noise_derivation}
%\end{equation}
As an intuitive example, if 
$P(n_l| \vec{b},\vec{x},a,\vec{y}) = \mathcal{N}(n_l; \mu; \sigma^2)$
is a simple Normal distribution with mean $\mu$ and variance $\sigma^2$ and we let $\alpha = 0.95$ 
then we know that that the 95\% of the probability mass
lies within $\mu \pm 2\sigma$, hence
\begin{equation*}
N(n_l| \vec{b},\vec{x},a,\vec{y}) = 
\begin{cases}
\mu - 2\sigma \leq n_l \leq \mu + 2 \sigma: & -\infty \text{ (legal)}\\
\text{otherwise} : & +\infty \text{ (illegal)}
\end{cases} \; .
\end{equation*}
%For more general state dependent distributions, ~\eqref{eq:noise_derivation} should
%be evaluated directly; fortunately when the interval constraints $C$ are linear,
%the linear-constrained integration techniques proposed by~\cite{sanner_aaai12} combined
%with the symbolic maximization approach of~\cite{sdp_aaai} often allow this noise function to be 
%computed analytically in closed-form when the indefinite integral over $n_l$ can
%be computed for $N(n_l| \vec{b},\vec{x},a,\vec{y})$ and gradient optimization
%is applied to the result to maximize constraint parameters.

To make the ND-HMDP framework concrete, we now introduce a running example used
throughout the paper:
%%%%%%%%%%%%%%%%%%%%%%%%% ZAHRA TODO
\begin{example*}[\textsc{Reservoir Control}]
The problem of maintaining maximal reservoir levels subject to
uncertain amounts of rainfall is an important problem in operations
research (OR) literature~\cite{Mahootchi2009,Yeh1985}.  In one
variant of this problem, a reservoir operator must
make a daily decision to \emph{drain} some water from a reservoir 
or not subject to weather forecasts over some time horizon.
Specifically in a seven day period, we assume that the weather forecast calls for a 
substantial amount of rain on the fourth day and chances of less rain on the
others.  The objective of the reservoir operator is to avoid underflow or
overflow conditions while maximizing reservoir capacity.

Formally, we assume a state consisting of continuous reservoir level $l_1 \in \mathbb{R}$
and 3 boolean variables $\vec{b}$ to encode a time period of eight days.  We have
two actions $a \in A = \{ \mathit{drain}, \mathit{no}-\mathit{drain} \}$.
The reward function $R$ is used to prevent overflow and underflow by assigning
$-\infty$ penalty to water levels outside of lower reserve and upper
capacity limits and a reward for the amount of water stored at the end of
the time step.  For both $a \in A$ this is formally defined as:
\vspace{-2mm} 
{\footnotesize
\begin{align*}
R(l_1,l_1',\vec{b},\vec{b}',a) &= 
\begin{cases}
(200 \! \leq \! l_1 \! \leq \! 4500) \wedge (200 \! \leq \! l'_1 \! \leq \! 4500) & \!\! : l_1'\\
\text{otherwise} & \!\! : -\infty\\
\end{cases}
\end{align*}
}
For the transition function, we assume that on each time step $\vec{b}' = \vec{b} + 1$ (not shown)
and the reservoir level changes according to the amount of outflow (2000 units of water on a
$\mathit{drain}$ and 0 units on a $\mathit{no}$-$\mathit{drain}$ action)
plus a noisy (uncertain) amount of rain $n$:
{\footnotesize
\begin{align*}
P(l'_1|l_1,n,d_i,d'_i, a& =\mathit{drain}) = \delta \left( l_1'  - (n + l_1 - 2000) \right) \\
P(l'_1|l_1,n,d_i,d'_i, a&=\mathit{no}-\mathit{drain}) =\delta \left( l_1'  -  (n + l_1) \right)
\end{align*}}
The use of the $\delta[\cdot]$ function here ensures that the continuous CPF over $l'$ integrates to 1,
which is crucial for defining a proper probability distribution.  While these PLEs are deterministic
note that all continuous noise in this framework enters via the non-deterministic noise variables
in ND-HMDPs.  The noisy level of rainfall $n$ is state-dependent and legal intervals are defined as follows:
{\footnotesize
\begin{align*}
N(n|\vec{b},l_1) = \begin{cases}
\vec{b} = 4 \wedge (1200 \leq n \leq 2000) &: -\infty \\
\vec{b} = 4  \wedge (0 \leq n \leq 400)&: -\infty \\
\text{otherwise} &: +\infty
\end{cases}
\end{align*}}
In short, on day four ($\vec{b} = 4$) the amount of rain is expected to be between
1200 and 2000 units, whereas on the other days it is expected to be between 0 and 400 units.
%%%%%%%%%%%%%%%%%%%%%%%%%%%%%
\end{example*}

A policy $\pi(\vec{b},\vec{x})$ specifies the action $a(\vec{y}) =
\pi(\vec{b},\vec{x})$ to take at state $(\vec{b},\vec{x})$.  
In a robust solution to HMDPs with non-deterministic noise constraints, 
an optimal sequence of finite horizon policies $\Pi^* = (\pi^{*,1},\ldots,\pi^{*,H})$ is desired such that given the initial state $(\vec{b}_0,\vec{x}_0)$ at $h=0$ and  a discount factor $\gamma, \; 0 \leq \gamma \leq 1$, the expected sum of discounted rewards over horizon $h \in H$ ($H \geq 0$) is maximized subject to Nature's
adversarial attempt to choose value minimizing assignments of the noise variables.  The
value function $V$ w.r.t. $\Pi^*$ in this case is defined via a recursive expectation
{\footnotesize
\begin{align*} 
& V^{\Pi^{*,H}}(\vec{b},\vec{x}) = \min_{\vec{n}} \, \max\!\Big(N(n_1|\vec{b},\vec{x},\Pi^{*,H}),\ldots,\\
& \max\!\Big(N(n_e|\vec{b},\vec{x},\Pi^{*,H}), E_{\Pi^{*,H}} \Big[ r^h \!\! + \! \gamma V^{\Pi^{*,H-1}} \! \! (\vec{b}',\!\vec{x}') \Big| \vec{b}_0,\!\vec{x}_0\Big] \Big) \!\! \cdots \!\! \Big)
\end{align*}}                   
where $r^h$ is the reward obtained at horizon $h$ following policy $\Pi^*$ and using
Nature's minimizing choice of $\vec{n}$ at each $h$.

The effect of ``max'ing'' in each of the previously defined
$N(n_l| \vec{b},\vec{x},a,\vec{y})$ ($1 \leq l \leq e$) with the value
function is one of the major insights and contributions of this paper.
We noted before that Nature will never choose an illegal value of
$n_l$ where $N(n_l| \vec{b},\vec{x},a,\vec{y}) = +\infty$, instead it
will choose a legal value of $n_l$ for which
$N(n_l| \vec{b},\vec{x},a,\vec{y}) = -\infty$ which when ``max'ed'' in
with the value function effectively vanishes owing to the indentity
$\max(v,-\infty) = v$ for all $v > -\infty$.

% B = 1 - H(1-\alpha)
% B - 1 = - H(1-\alpha)
% (B - 1) / -H = 1-\alpha
% (B - 1) / -H - 1 = -\alpha
% 1 + (B - 1) / H = \alpha
% 1 - (1 - B) / H = \alpha
Finally, by leveraging the simple union bound, we can easily prove that
that a policy will achieve $V^{\Pi^{*,H}}$ with at least $1 - H(1-\alpha)$
probability since the probability of encountering a noise value
outside the confidence interval is only $(1-\alpha)$ at any time step.
Hence for a success probability of at least $\beta$, one should choose
$\alpha = 1 - \frac{1 - \beta}{H}$, e.g., $\beta = 0.95$ success probability
requires an $\alpha = 0.99$ for $H=5$.

 
\subsection{Robust Dynamic Programming}

We extend the value iteration dynamic programming
algorithm~\cite{bellman} and specifically the form used for HMDPs
in~\cite{sdp_aaai} to a robust dynamic programming (RDP) algorithm for
ND-HMDPs that may be considered a continuous action generalization of
zero-sum alternating turn Markov games~\cite{littman94}.  Initializing
$V^0(\vec{b},\vec{x}) = 0$ the algorithm builds the $h$-stage-to-go
value function $V^h(\vec{b},\vec{x})$.

The quality $Q_a^{h}(\vec{b},\vec{x},\vec{y},\vec{n})$ of taking
action $a(\vec{y})$ in state $(\vec{b},\vec{x})$ with noise parameters
$\vec{n}$ and acting so as to obtain $V^{h-1}(\vec{b}',\vec{x}')$ thereafter 
is defined as the following:
%\vspace{-4mm}
{\footnotesize
\begin{align}
Q_a^{h}&(\vec{b},\vec{x},\vec{y},\vec{n}) \! = \!
\max\!\Big(\!\!N(n_1|\vec{b},\vec{x},\Pi^{*,H}),\!\ldots\!, \max\!\Big(\!\!N(n_e|\vec{b},\vec{x},\Pi^{*,H}), \nonumber \\
& 
\sum_{\vec{b}'} \int \prod_{i=1}^a P(b_i'|\vec{b},\vec{x},\vec{b}',\vec{x}',a,\vec{y},\vec{n}) \!
\prod_{j=1}^c P(x_j'|\vec{b},\vec{x},\vec{b}',\vec{x}',a,\vec{y},\vec{n}) \! \nonumber \\
& \qquad \qquad \; \Big[ R(\vec{b},\vec{x},\vec{b}',\vec{x}',a,\vec{y}) + \gamma V^{h-1}(\vec{b}',\vec{x}') d\vec{x}' \Big] \Big) \! \cdots \! \Big) \hspace{-2mm} \label{eq:qfun}
\end{align}}
Here the noise constraints $N(\vec{n},\vec{b},\vec{x})$ are ``max'ed''
in with the value function to ensure Nature chooses a legal setting 
of $n_l$, effectively reducing each $\max$ to an identity operation.  

Next, given $Q_a^h(\vec{b},\vec{x},\vec{y},\vec{n})$ as above for each
$a \in A$, we can proceed to define the $h$-stage-to-go value function
assuming that the agent attempts to maximize value subject to Nature's
adversarial choice of value-minimizing noise:
\vspace{-2mm}
{\footnotesize
\begin{align}
V^{h}(\vec{b},\vec{x}) & = \max_{a \in A} \max_{\vec{y} \in \mathbb{R}^{|\vec{y}|}} \min_{\vec{n} \mathbb{R}^{|\vec{e}|}} \left\{ Q^{h}_a(\vec{b},\vec{x},\vec{y},\vec{n}) \right\} \label{eq:vfun}
\end{align}}
The optimal policy at horizon $h$ can also be determined using the
$Q$-function as below:
{\footnotesize
\begin{align}
\pi^{*,h}(\vec{b},\vec{x}) = \argmax_{a \in A}  \argmax_{\vec{y} \in \mathbb{R}^{|\vec{y}|}}  \min_{\vec{n} \mathbb{R}^{|\vec{e}|}}  Q^h_a(\vec{b},\vec{x},\vec{y},\vec{n})
\end{align}}
For finite-horizon HMDPs the optimal value function and policy are
obtained up to horizon H.  For infinite horizons where the optimal policy
has finitely bounded value then value iteration terminates when two
values are equal in subsequent horizons ($V^{h} = V^{h-1}$). In this
case $V^\infty = V^h$ and $\pi^{*,\infty} = \pi^{*,h}$.
 
Up to this point we have only provided the abstract mathematical
framework for ND-HMDPs and RDP.  Fortuitously though, we can leverage
the continuous $\max$ (and analoguously defined $\min$) operations and
symbolic DP approach of~\cite{sdp_aaai} in order to compute RDP 
via~\eqref{eq:qfun} and \eqref{eq:vfun} exactly in closed-form.  We
discuss this next.
