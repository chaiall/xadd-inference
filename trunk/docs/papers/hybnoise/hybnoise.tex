\documentclass{article}
% The file ijcai13.sty is the style file for IJCAI-13 (same as ijcai07.sty).
\usepackage{ijcai13}

% Use the postscript times font!
\usepackage{times}

% See ../linquad.cameraready/exact_sdp.tex for AAAI-12 Latex source
\title{	Approximate Symbolic Dynamic Programming for Hybrid MDPs }
\author{Luis Gustavo Rocha Vianna\\
University of Sao Paulo\\
Sao Paulo, Brazil \\
ludygrv@gmail.com
\And
Scott Sanner \\
NICTA / ANU\\
Canberra, Australia \\
ssanner@nicta.com.au
\And
Leliane Nunes de Barros\\
University of Sao Paulo\\
Sao Paulo, Brazil\\
leliane@ime.usp.br}

\begin{document}

\maketitle

\begin{abstract}
Recent advances in approximate symbolic dynamic programming (SDP) have 
provided exact solutions for mixed discrete and continuous (hybrid) 
MDPs with piecewise linear dynamics and continuous actions. Combined 
with the extended algebraic decision diagram (XADD) data structure to 
compactly represent functions over discrete and continuous variables, 
approximate SDP methods have been able to solve problems such as 
multivariate inventory control for which exact solutions were 
previously considered intractable. Unfortunately, XADD-based 
solutions to these problems can still grow quite large as problem size 
increases leading to the natural question of whether XADDs can be 
approximately compressed within fixed error bounds. To this end, we 
propose a compression technique for linear XADDs that requires the 
repeated solution of a bilinear program with an exponential number of 
constraints. Fortuitously, we show that this class of bilinear 
programs can be rewritten as an iterative bilevel linear programming 
problem that mitigates the need to generate an exponential number of 
constraints while still guaranteeing optimality. This solution 
permits the use of efficient linear program solvers and enables a 
novel class of bounded approximate SDP algorithms based on XADD 
compression. Empirically, we demonstrate this approach offers 
order-of-magnitude speedups over the exact solution for a variety of 
hybrid MDPs in exchange for relatively small approximation error.
\end{abstract}

\section{Introduction}

\cite{sanner_uai11}

%\section*{Acknowledgments}

%\appendix

%% The file named.bst is a bibliography style file for BibTeX 0.99c
\bibliographystyle{named}
\bibliography{hybnoise}

\end{document}

