%%HYBRID SYSTEMS

Hybrid systems are a class of dynamical systems that involve both continuous and discrete dynamics. The dynamics of the continuous variables are
defined typically through differential equations and the evolution of the discrete variables through finite state machines, Petri nets or other abstract computational machines. One accepted manner to model hybrid systems is using hybrid automata that represents, in a single formalism, the discrete changes by automata transitions and the continuous changes by differential equations \cite{DeSHee:2009,Henzinger:1997}.
One special class of hybrid systems are the switched linear systems that have a collection of subsystems defined by linear dynamics (differential equations) and a switching rule that specifies the switching between the subsystems \cite{Sun:2005}.

One problem that has been studied in the area of hybrid systems is the verification of the safety property, that tries to proof that the system does not enter in unsafe configurations from an initial configuration \cite{Tomlin:2003}. Then, we say that the system satisfies the safety property if all reachable states are safe \cite{Henzinger:1997}. There are many tools for the automatic verification of hybrid systems such as HyTech \cite{Henzinger:1997}, KRONOS \cite{Yovine:1997}, PHAVer \cite{Frehse:05} and HSOLVER \cite{Ratschan:2007}. All the techniques rely on the ability to compute reachable sets of hybrid systems. For example, HyTech, a symbolic model checker, automatically computes reachable sets for linear hybrid automata, a subclass of hybrid automata. HyTech can also return the values of design parameters for which this automata satisfies a temporal-logic requirement \cite{Henzinger:1997}. Some examples of verification of hybrid systems can be found in \cite{Henzinger:1997,Mohrenschildt:2001}.

Another challenging topic in hybrid systems is to evaluate the effect of the hybrid controller on the systems operation, i.e., to solve the controllability problem for hybrid systems \cite{Stikkel:2004}. A hybrid system is called hybrid controllable if, for any pair of valid states, there exists at least one permitted control sequence (correct control-laws) between them \cite{Tittus:1998,Yang:2007}. The general controllability problem of hybrid systems is NP hard \cite{Blondel:1999}. However, for special classes of hybrid systems, some necessary and sufficient conditions for controllability were obtained in \cite{Stikkel:2004,Lemch:2001,Sun:2002,Zhenyu:2003,Yang:2007}.
For example, by employing algebraic manipulation of system matrices, a sufficient and necessary condition for the controllability analysis of a class of  piecewise linear hybrid systems is given in {Zhenyu:2003}. This class is called controlled switching linear hybrid system and have the following properties: all mode switches are controllable, the dynamical subsystems within each mode has a LTI form, the admissible operating regions within each mode is the whole state space, and there are no discontinuous state jumps. The controllability test for this class of hybrid system can be determined based on the system matrices. In \cite{Yang:2007} is proposed an approach for controllability analysis of a class of more complex hybrid systems. This approach uses a discrete-path searching algorithm that integrates global reachability analysis at the discrete event system level and a local reachability analysis at the continuous level. This method cannot guarantee the existence of a solution for an arbitrary hybrid system \cite{Yang:2007}.

Much of the work on hybrid systems has focused on deterministic models without allowing any uncertainty. In practice, there are real world applications where the environment is inherent uncertainty. To cope with this, the stochastic hybrid systems was proposed. Stochastic hybrid systems allow uncertainty (1) replacing deterministic jumps between discrete states by random jumps or (2) replacing the deterministic dynamics inside the discrete state by a stochastic differential equation or (3) combinations of 1 and 2 \cite{Hu:2000}. A critical problem in this type of systems is the verification of reachability properties because it is necessary to cope with the interaction between the discrete and continuous stochastic dynamics, in this case it is computed the probability that the system satisfies the property \cite{Koutsoukos:2006}.  Related with the concept of verification of safety property, in stochastic hybrid systems, the system tries to maximize the probability that the execution will remain in safe states as long as possible \cite{Hu:2000}.

%%CHANCE-CONSTRAINED CONTROL

Chance-constrained predictive stochastic control of  dynamic systems  characterizes uncertainty in a probabilistic manner, and finds the optimal sequence of control inputs subject to the constraint that the probability of failure must be below a user-specified threshold \cite{Blackmore:2011}. This constraint is known as a chance constraint \cite{Blackmore:2011} and is used to define stochastic robustness.

A great deal of work has taken place in recent years relating to chance-constrained optimal control of linear systems subject to Gaussian uncertainty in convex regions \cite{Schwarm:1999,Li:2002,Ono:2008} and linear systems in nonconvex regions \cite{Blackmore:2010,Blackmore:2011}.
The approach in \cite{Blackmore:2010} uses samples, or ‘particles’, to approximate the  chance constraint, and hence does not guarantee satisfaction of the constraint. It applies to arbitrary uncertainty distributions and is significantly more computationally intensive than others. The approach proposed in \cite{Blackmore:2011} uses analytic bound to ensure satisfaction of the constraint and applies for linear-Gaussian systems.

