In order to compute the equations above, we propose a \emph{robust
symbolic dynamic programming} (RSDP) approach building on the
work of~\cite{sdp_aaai,sanner_uai11}.  This requires a value iteration
algorithm described in Algorithm~\ref{alg:vi} (\texttt{VI}) and the
regression subroutine described in Algorithm~\ref{alg:regress}.  In what
follows we show how the techniques of SDP can be extended to compute RDP
exactly in closed-form as discussed in the last section.

In general we define \emph{all} symbolic functions to be represented
in \emph{case} form~\cite{fomdp} for which a binary ``cross-sum'' operation
can be defined as follows:
{\footnotesize 
\begin{center}
\begin{tabular}{r c c c l}
&
\hspace{-6mm} 
  $\begin{cases}
    \phi_1: & f_1 \\ 
    \phi_2: & f_2 \\ 
  \end{cases}$
$\oplus$
&
\hspace{-4mm}
  $\begin{cases}
    \psi_1: & g_1 \\ 
    \psi_2: & g_2 \\ 
  \end{cases}$
&
\hspace{-2mm} 
$ = $
&
\hspace{-2mm}
  $\begin{cases}
  \phi_1 \wedge \psi_1: & f_1 + g_1 \\ 
  \phi_1 \wedge \psi_2: & f_1 + g_2 \\ 
  \phi_2 \wedge \psi_1: & f_2 + g_1 \\ 
  \phi_2 \wedge \psi_2: & f_2 + g_2 \\ 
  \end{cases}$
\end{tabular}
\end{center}
}
\normalsize
Here $\phi_i$ and $\psi_j$ are logical formulae defined over the state
$(\vec{b},\vec{x})$ and can include arbitrary logical
($\land,\lor,\neg$) combinations of boolean variables
and \emph{linear} inequalities ($\geq,>,\leq,<$) over continuous
variables -- we call this \emph{linear case form} (LCF).  The $f_i$
and $g_j$ are restricted to be \emph{linear} functions.  Similarly
operations such as $\ominus$ and $\otimes$ may be defined with operations
applied to LCF functions yielded LCF results.

In addition to $\ominus$ and $\otimes$ another key binary operation on case
statements the preserves the LCF property is \emph{symbolic case maximization}:
\vspace{-7mm}
{\footnotesize
\begin{center}
\begin{tabular}{r c c c l}
&
\hspace{-7mm} $\casemax \Bigg(
  \begin{cases}
    \phi_1: \hspace{-2mm} & \hspace{-2mm} f_1 \\ 
    \phi_2: \hspace{-2mm} & \hspace{-2mm} f_2 \\ 
  \end{cases}$
$,$
&
\hspace{-4mm}
  $\begin{cases}
    \psi_1: \hspace{-2mm} & \hspace{-2mm} g_1 \\ 
    \psi_2: \hspace{-2mm} & \hspace{-2mm} g_2 \\ 
  \end{cases} \Bigg)$
&
\hspace{-4mm} 
$ = $
&
\hspace{-4mm}
  $\begin{cases}
  \phi_1 \wedge \psi_1 \wedge f_1 > g_1    : & \hspace{-2mm} f_1 \\ 
  \phi_1 \wedge \psi_1 \wedge f_1 \leq g_1 : & \hspace{-2mm} g_1 \\ 
  \phi_1 \wedge \psi_2 \wedge f_1 > g_2    : & \hspace{-2mm}f_1 \\ 
  \phi_1 \wedge \psi_2 \wedge f_1 \leq g_2 : & \hspace{-2mm} g_2 \\ 
 \vdots & \vdots
 %\phi_2 \wedge \psi_1 \wedge f_2 > g_1    : & \hspace{-2mm} f_2 \\ 
% \phi_2 \wedge \psi_1 \wedge f_2 \leq g_1 : & \hspace{-2mm} g_1 \\ 
% \phi_2 \wedge \psi_2 \wedge f_2 > g_2    : & \hspace{-2mm} f_2 \\ 
% \phi_2 \wedge \psi_2 \wedge f_2 \leq g_2 : & \hspace{-2mm} g_2 \\ 
  \end{cases}$
\end{tabular}
\end{center}
} \vspace{-1mm}

%%%%%%%%%%%%%%%%%%%%%%%%%%%%%%%%%%%%%%%%%%%%%%%%%%%%%%%%%%%%%%%%%%%%%%%%
\incmargin{.5em}
\linesnumbered
\begin{algorithm}[t!]
\vspace{-.5mm}
\dontprintsemicolon
\SetKwFunction{regress}{Regress}
\Begin{
   $V^0:=0, h:=0$\;
   \While{$h < H$}{
       $h:=h+1$\;
       \ForEach {$a(\vec{y}) \in A$}{
              $Q_a^{h}(\vec{y},\vec{n})\,:=\,$\regress{$V^{h-1},a,\vec{y}$}\;
				$Q_a^{h}(\vec{y}) := \min_{\vec{n}} \, Q_a^{h}(\vec{y},\vec{n})$ $\,$\emph{//Stochastic $\min$}\;
				$Q_a^{h} := \max_{\vec{y}} \, Q_a^{h}(\vec{y})$ $\,$ \emph{// Continuous $\max$}\;
              $V^{h} := \casemax_a \, Q_a^{h}$ $\,$ \emph{// $\casemax$ all $Q_a$}\;
              $\pi^{*,h} := \argmax_{(a,\vec{y})} \, Q_a^{h}(\vec{y})$\;
       }
       \If{$V^h = V^{h-1}$}
           {break $\,$ \emph{// Terminate if early convergence}\;}
   }
     \Return{$(V^h,\pi^{*,h})$} \;
}
\caption{\footnotesize \texttt{VI}(CSA-MDP, $H$) $\longrightarrow$ $(V^h,\pi^{*,h})$ \label{alg:vi}}
\vspace{-1mm}
\end{algorithm}
\decmargin{.5em}
%%%%%%%%%%%%%%%%%%%%%%%%%%%%%%%%%%%%%%%%%%%%%%%%%%%%%%%%%%%%%%%%%

%%%%%%%%%%%%%%%%%%%%%%%%%%%%%%%%%%%%%%%%%%%%%%%%%%%%%%%%%%%%%%%%%
\incmargin{.5em}
\linesnumbered
\begin{algorithm}[t!]
\vspace{-.5mm}
\dontprintsemicolon
\SetKwFunction{remapWithPrimes}{Prime}
%\SetKwFunction{sumout}{sumout}

\Begin{
    $Q=$ \remapWithPrimes{$V$} $\,$ \emph{// All $b_i \to b_i'$ and all $ x_i \to x_i'$} \;
%%%%%%%%%%%%%%%%%%% ZAHRA TODO ( check casemax of noise is correct) 
 	\If {$v'$ in $R$}
	 {$Q := R(\vec{b},\vec{b}',\vec{x},\vec{x}',a,\vec{y}) \oplus (\gamma \cdot Q)$} \;
    \ForEach { $v'$ in $Q$}  
    {
    	\If {$v'$ = $x'_j$}
    	{
    	\emph{//Continuous marginal integration}\\
         $Q := \int Q \otimes P(x_j'|\vec{b},\vec{b}',\vec{x},\vec{x}',a,\vec{y},\vec{n}) \, d_{x'_j}$\;
    	}
	    \If {$v'$=$b'_i$}
    	{
    	\emph{// Discrete marginal summation}\\
         $Q := \left[ Q \otimes P(b_i'|\vec{b},\vec{b}',\vec{x},\vec{x}',a,\vec{y},\vec{n}) \right]|_{b_i' = 1}$\\
         \hspace{8mm} $\oplus \left[ Q \otimes P(b_i'|\vec{b},\vec{b},\vec{x},\vec{x}',a,\vec{y},\vec{n}) \right]|_{b_i' = 0}$\;
    	}
    }
    \If {$\neg$ ($v'$  in $R$)}
    {$Q := R(\vec{b},\vec{b}',\vec{x},\vec{x}',a,\vec{y}) \oplus (\gamma \cdot Q)$ }\;
     \ForEach { $n_l$ in $Q$}
     {
		\emph{// Sequence of $\max$-in for noise variables}\\          
         $Q_a^{h}(\vec{y},\vec{n}) := \casemax_{n_l} \, ( Q, N(n_l, b_i,x_j))$ $\,$ \;
	}
	%%%%%%%%%%%%%%%%%%%%
    \Return{$Q$} \;
}
\caption{\footnotesize \texttt{Regress}($V,a,\vec{y}$) $\longrightarrow$ $Q$ \label{alg:regress}}
\vspace{-1mm}
\end{algorithm}
\decmargin{.5em}
%%%%%%%%%%%%%%%%%%%%%%%%%%%%%%%%%%%%%%%%%%%%%%%%%%%%%%%%%%%%%%%%%

To demonstrate how \texttt{VI} symbolically implements RDP, we compute $V^1$ for the \textsc{Reservoir Control} example. For both actions, the function $Q^1_a$ is computed in line 6 using Algorithm~\ref{alg:regress} with the following operations for action $\mathit{no}$-$\mathit{drain}$:
\begin{itemize}

\item Priming V which %the current state variables ($d_i,l_j$) to build the next states ($d_i',l_j'$) in the value function. This
indicates a \emph{symbolically substitution} of  $V'^0= V^0 \sigma = 0$ where $\sigma = \lbrace d_i \setminus d_i' , l_j \setminus l_j' \rbrace$.% for all values of $i$ and $j$. 

\item Since the reward function contains the primed variable $l_1'$, line 4 is performed (and not line 15) where $Q = R(l_1,l'_1,d_i,d'_i,n,a) $. 

\item For boolean variables, regression is performed using $f|_{b=v}$ (restriction operator) which assigns the value $v \in \{ 0,1 \}$ to any occurrence of $b$ in $f$-- not applicable to the example. For continuous variables line 9 follows the rules of integration w.r.t. a $\delta$ function~\cite{sanner_uai11} which simply yields a symbolic substitution: 
{\footnotesize
\begin{align}
\int f(x'_j) \otimes \delta[x_j' - h(\vec{z})] dx'_j = f(x'_j) \{ x'_j / h(\vec{z}) \}\nonumber
\end{align}}
This results in the following $Q$-value for \textsc{Reservoir Control} : 
{\footnotesize
\begin{align*}
%Q =   
\begin{cases}
(200 \leq l_1 \leq 4500) \wedge (200 \leq (l_1 + n) \leq 4500) &: \hspace{-1mm} l_1 + n\\
\text{otherwise} &: -\infty\\
\end{cases}
\end{align*}
}
\item Maximizing the result with each of the noise variables in defined line 20 using a sequence of symbolic maximizations. Each noise variable assigns -$\infty$ for legal values inside the boundary range +$\infty$ for illegal values defined by the noise model $N(\vec{n},\vec{b},\vec{x})$. The result is defined below:
%%%%%%%%%%%%%%%% Not in the log file and too big to fit
%{\footnotesize
%\begin{align}
%\begin{cases}
%1200 \leq n \leq 2000 :& -\infty \\
%\neg(1200 \leq n \leq 2000) :& +\infty \\
%d_1 \land (200 \leq l_1 \leq 4500) \land (1200 \leq n \leq 2000) \land (200 \leq (l_1+n) \leq 4500): & l_1 +n \\ 
%d_1 \land (200 \leq l_1 \leq 4500) \land (1200 \leq n \leq 2000) \land \neg (200 \leq (l_1+n) \leq 4500): & - \infty \\ 
%d_1 \land \neg (200 \leq l_1 \leq 4500) \land (1200 \leq n \leq 2000) : & -\infty \\  
%d_1 \land (200 \leq l_1 \leq 4500) \land \neg(1200 \leq n \leq 2000) : & +\infty \\  
%d_1 \land \neg (200 \leq l_1 \leq 4500) \land \neg(1200 \leq n \leq 2000) : & +\infty \\  
%0 \leq n \leq 400 :& -\infty \\
%\neg(0 \leq n \leq 400) :& +\infty \\
%\neg d_1 \land (200 \leq l_1 \leq 4500) \land (0 \leq n \leq 400) \land (200 \leq (l_1+n) \leq 4500): & l_1 +n \\ 
%\neg d_1 \land (200 \leq l_1 \leq 4500) \land (0 \leq n \leq 400) \land \neg (200 \leq (l_1+n) \leq 4500): & - \infty \\ 
%\neg d_1 \land \neg(200 \leq l_1 \leq 4500) \land (0 \leq n \leq 400) : & -\infty \\  
%\neg d_1 \land (200 \leq l_1 \leq 4500) \land \neg(0 \leq n \leq 400) : & +\infty \\  
%\neg d_1 \land \neg(200 \leq l_1 \leq 4500) \land \neg(0 \leq n \leq 400) : & +\infty \\  
%\end{cases} \nonumber
%\end{align}
%}
%%%%%%%%%%%%%%%%%%%%%
{\footnotesize
\begin{align*}
%Q =   
\begin{cases}
((l_1 \land (l_1+n)) \in \mathit{safe}) \land (n \in \mathit{legal}) &: l_1 + n\\
(l_1  \in \mathit{safe}) \land  ((l_1+n) \notin \mathit{safe}) \land (n \in \mathit{legal}) &:  -\infty\\
(n \notin \mathit{legal})&: +\infty\\
\end{cases}
\end{align*}
}
where $\mathit{legal}$ noise value corresponds to [0,400] or [1200,2000] and $\mathit{safe}$ water levels is [200,4500]. 
%The result assigns the value of $l_1+n$ to both safe water levels and legal noise ranges. Unsafe current or next state water levels (with legal noise) lead to -$\infty$ and illegal noise ranges cause the value to be +$\infty$.

The regressed stochastic $Q_a^{h}(\vec{y},\vec{n})$ from Algorithm~\ref{alg:regress} is now minimized over the noise variables $\vec{n}$ in line 7. Intuitively this continuous minimization will never choose +$\infty$ as there is always some value smaller which insures that the transitioned model never chooses illegal values. Each partition $i$ of this intermediate $Q$ is considered for a continuous minimization separately with the final result a $\casemin$ (definition follows from $\casemax$) on all the individual minimum results: $\casemin_i \min_n \phi_i(\vec{b},\vec{x},\vec{n}) f_i(\vec{b},\vec{x},\vec{n})$.  
%%%%%%%%%%%%%%%%%% TODO: Can take this piece out completely
We demonstrate the steps of this algorithm for the third partition of the regressed $Q$ defined as:
{\footnotesize
\begin{align*}
d_1 \land (200\leq l_1\leq 4500) &\land (1200\leq n\leq 2000) \\
& \land (200\leq (l_1+n)\leq 4500): l_1 +n 
\end{align*}}
For each partition the logical constraints are used to derive the (a) lower bound on $n$ ($LB =$ 1200, $200 -l_1$); (b) upper bound on $n$ ($UB =$ 2000, $4500- l_1$)  and (c) constraints independent of $n$ ($IND= d_1, 200 \leq l_1 \leq 4500$). In case of several bounds on $n$ the maximum of all lower bounds and the minimum of all upper bounds is desired.:
{\footnotesize
\begin{center}
\begin{tabular}{r c c c l}
&
$
LB =  
\begin{cases}
 l_1 < -1000 : & 200 - l_1 \\ 
 l_1 > -1000 : & 1200 \\ 
\end{cases} 
$
\\
&
$ UB =  
\begin{cases}
l_1 >2500 : & 4500 - l_1 \\ 
l_1<2500 : & 2000 \\ 
\end{cases} 
$
\end{tabular}
\end{center}
}
The minima points of upper and lower bounds are evaluated for the leaf value which equals to substituting the bounds instead of the noise variable $n$ in the leaf function $n+ l_1$:
{\footnotesize
\begin{align}
Q =  
\begin{cases}
 l_1\leq - 1798 :&  2000 + l_1 \\
 l_1\leq - 1000 :&  200 \\
 l_1\leq  3300 :&  1200.48 + l_1 \\
  l_1\geq  3300 :&  4500 \\
\end{cases} \nonumber
\end{align}
}
Natural constraints on bounds $\LB \leq \UB$ and the $IND$ constraints are also considered for the minimization on a single partition to obtain: 
{\footnotesize
\begin{align}
Q =  
\begin{cases}
d_1 \land ( 2000 \leq l_1 \leq 3300 ):&  1200.48 + l_1 \\
\text{otherwise} :&  + \infty \\ 
\end{cases} \nonumber
\end{align}
}
%%%%%%%%%%%%%%%%%End of take this out
The final result of a continuous minimization is a $\casemin$ over all partitions which results in the following Q-value:
{\footnotesize
\begin{align}
Q^1_{\mathit{no}-\mathit{drain}} =  
\begin{cases}
d_1 \land (200 \leq l_1 \leq 2506) : & 1200.48 + l_1 \\
\neg d_1 \land (200 \leq l_1 \leq 4098) : & l_1 \\
\text{otherwise} : & -\infty \\ 
\end{cases} \nonumber
\end{align}
}

The resulting Q-value with minimal noise is maximized over the continuous action parameter (not available in our example) in line 8. A discrete $\casemax$ on the set of discrete actions for all $Q$-functions defines the final $V$:
{\footnotesize
\begin{align}
V^1 =  
\begin{cases}
d_1 \land (2506 \leq l_1 \leq 4484): & 1 \\
d_1 \land (200 \leq l_1 \leq 2506): & 1200.48 + l_1 \\
\neg d_1 \land (4098 \leq l_1 \leq 4504) : & 1 \\
\neg d_1 \land (200 \leq l_1 \leq 4098) : & l_1 \\
\text{otherwise} : & -\infty \\ 
\end{cases} \nonumber
\end{align}
}
\end{itemize}

To implement the case statements efficiently with continuous variables, extended Algebraic Decision diagrams (XADDs) are used from ~\cite{sanner_uai11} extended from ADDs ~\cite{bahar93add}. Unreachable paths can be pruned in XADDs using LP solvers.  We crucially note that all operations including the continuous maximization and minimization preserve the LCF property, hence all operations for robust SDP can be performed exactly in closed-form -- a first for receding horizon control with general forms of state-dependent continuous noise.
