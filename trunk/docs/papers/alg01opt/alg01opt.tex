%%%%%%%%%%%%%%%%%%%%%%%%%%%%%%%%%%%%%%%%%%%%%%%%%%%%%%%%%%%%%%%%%%
%%%%%%%% ICML 2013 EXAMPLE LATEX SUBMISSION FILE %%%%%%%%%%%%%%%%%
%%%%%%%%%%%%%%%%%%%%%%%%%%%%%%%%%%%%%%%%%%%%%%%%%%%%%%%%%%%%%%%%%%

% Use the following line _only_ if you're still using LaTeX 2.09.
%\documentstyle[icml2013,epsf,natbib]{article}
% If you rely on Latex2e packages, like most moden people use this:
\documentclass{article}

% For figures
\usepackage{graphicx} % more modern
%\usepackage{epsfig} % less modern
\usepackage{subfigure} 
\usepackage{enumerate}
\usepackage{amsmath}
\usepackage{amssymb}
\usepackage{framed}
\usepackage{url}
\usepackage{multirow}
%\usepackage{algorithm}
\usepackage{algpseudocode}
%\usepackage{algorithmic}


% For citations
\usepackage{natbib}

% For algorithms

% As of 2011, we use the hyperref package to produce hyperlinks in the
% resulting PDF.  If this breaks your system, please commend out the
% following usepackage line and replace \usepackage{icml2013} with
% \usepackage[nohyperref]{icml2013} above.
\usepackage{hyperref}

% Packages hyperref and algorithmic misbehave sometimes.  We can fix
% this with the following command.
\newcommand{\theHalgorithm}{\arabic{algorithm}}

%\COMMENT begins comment
%\ENDCOMMENT ends comment
\long\def\COMMENT#1\ENDCOMMENT{\message{(Commented text...)}\par}

% Employ the following version of the ``usepackage'' statement for
% submitting the draft version of the paper for review.  This will set
% the note in the first column to ``Under review.  Do not distribute.''
\usepackage{icml2013} 
% Employ this version of the ``usepackage'' statement after the paper has
% been accepted, when creating the final version.  This will set the
% note in the first column to ``Proceedings of the...''
% \usepackage[accepted]{icml2013}

\def\R{\mathbb{R}}
\def\S{\mathbb{S}}
\def\xi{\boldsymbol{x}_i}
\def\x{\boldsymbol{x}}
\def\w{\boldsymbol{w}}
\def\p{\boldsymbol{p}}
\def\l{\boldsymbol{l}}
\def\t{\boldsymbol{t}}
\def\X{\boldsymbol{X}}
\def\D{\boldsymbol{D}}

\newcommand{\BB}{BnB}
\newcommand{\sign}{\operatorname{sign}}

\def\argmax{\operatornamewithlimits{arg\,max}}
\def\argmin{\operatornamewithlimits{arg\,min}}

% The \icmltitle you define below is probably too long as a header.
% Therefore, a short form for the running title is supplied here:
\icmltitlerunning{Algorithms for Direct 0--1 Loss Optimization in Binary Classification}

\begin{document} 

\twocolumn[
\icmltitle{Algorithms for Direct 0--1 Loss Optimization in Binary Classification}

% It is OKAY to include author information, even for blind
% submissions: the style file will automatically remove it for you
% unless you've provided the [accepted] option to the icml2013
% package.
\icmlauthor{Tan T. Nguyen}{tan1889@gmail.com}
\icmladdress{ANU, Canberra, ACT 0200}
\icmlauthor{Scott P. Sanner}{ssanner@nicta.com.anu}
\icmladdress{NICTA \& ANU, Canberra, ACT 2601}

% You may provide any keywords that you 
% find helpful for describing your paper; these are used to populate 
% the "keywords" metadata in the PDF but will not be shown in the document
\icmlkeywords{0-1 loss, classification, optimization}

\vskip 0.3in
]

\begin{abstract} 
\input abstract
\end{abstract} 

%% Main contents
\input intro
\input background
\input bandb
\input combsearch
\input smoothloss
\input empirical
\input conclusions

% Bibliography
\bibliography{alg01opt}
\bibliographystyle{icml2013}

\end{document} 


