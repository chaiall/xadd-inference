\section{Branch and Bound}
\label{cha:branchandbound}

Branch and bound techniques (\BB)~\cite{bandb} are a staple of the
literature on discrete optimization via search methods.  This section
shows how to formulate the 0--1 loss problem in a \BB\ setting along
with heuristics and pruning techniques that allow it to efficiently
find an optimal 0--1 loss solution.

Following from the definition of the training objective
from~\eqref{eq:objective} w.r.t. 0--1 loss in~\eqref{eq:loss01},
the 0--1 loss optimization problem can be written as
\begin{equation}
L(\w) = \sum_{i=1}^N \mathbb{I} [t_i \w^T \xi \leq 0] = \sum_{i=1}^N l_i, \label{eq:loss01_data}
\end{equation}
where $l_i =\mathbb{I} [t_i \w^T \xi \leq 0]$ denotes the individual
0--1 loss of point $\xi$.  Then we have
\[ \begin{split}
l_i = 0 &\Leftrightarrow t_i \w^T \xi > 0 \\
l_i = 1 &\Leftrightarrow t_i \w^T \xi \leq 0. \\
\end{split} \] 
Furthermore, let $\l=(l_1, l_2, \dots, l_N)$ be the \emph{loss vector}
consisting of all individual losses. It can be seen that a specific
assignment of the loss vector corresponds to a system of linear
inequalities. For example, the loss vector $\l=(0,1,0)$ for training
data of $N=3$ points provides 
$\{ t_1 \w^T \x_1 > 0, t_2 \w^T \x_2 \leq 0, t_3 \w^T \x_3 > 0 \}$, 
where $t_1, t_2, t_3, \x_1, \x_2, \x_3$ are constants given by the
data, resulting in a system of linear inequalities.

Thus, the 0--1 loss optimization problem is now equivalent to finding
a feasible loss vector $\l^*$ with a minimal sum of its components
(which is the sum of 0--1 losses):
\begin{equation}
\l^* = \text{arg min}_{\l : \text{ feasible}} \sum_{i=1}^N l_i. \label{eq:bb_obj}
\end{equation}
Once we have obtained $\l^*$, we need to recover optimal hyperplane
parameters $\w^*$, which are insensitive to scaling.  Hence, we fix
$w_0$ at either $1$ or $-1$\footnote{$w_0 = 0$ would imply the optimal
  hyperplane goes through the origin, which is highly unlikely in
  practice.  Nonetheless $w_0 = 0$ could also be tested for
  completeness.} (whichever yields a feasible solution) and call a
linear program (LP) solver to minimize $\sum_{i=1}^D w_i$ 
(to obtain a unique solution) subject to the constraints induced by $\l^*$.

%=================================================
\begin{figure}[t!]
\vspace{-3mm}
\caption{
Branch and Bound Search (BnB).\hfill\; \\
\text{\hspace{1.4cm}} $Input$: Training data $(\boldsymbol{X},\t)$. \\
\text{\hspace{1.4cm}} $Output$: Weights $\w^*$ minimizing 0--1 loss.
}
\label{alg:BnB.Final}
{\footnotesize
\begin{algorithmic}[1]
\Function{Find-Optimal-01Loss-BnB}{$\X, \t$}
\State $\tilde{\w} \gets \w^*_{\mathit{SVM}}$ from SVM solution for $(\boldsymbol{X},\t)$
\State $\tilde{\l} \gets $ loss vector implied by $\tilde{\w}$
\State $\w^* \gets \tilde{\w}$
\State $loss_{min} \gets \sum_{i=1}^N \tilde{l}_i$ \Comment{Set initial bound to $L(\tilde{\w})$}
\State $\l_\emptyset \gets $ all decisions $l_i$ unassigned
\State \Call{Branch-and-Bound}{$\l_\emptyset, 0$}
\State \Return $\w^*$
\Statex
\Procedure{Branch-and-Bound}{$\l, loss$}
   \If {(all components of $\l$ are assigned)}
      \State $\w^* \gets$ LP solution for $\l$ 
      \State $loss_{min} \gets loss$
   \Else
      \State $i \gets \argmax_{i \text{unassigned} \in \l} |\tilde{\w}^Tx_i|$
      \State $\l' \gets$ \Call{Propagate-Loss}{$\l, i, \tilde{l}_i$}
      \If {$\sum_i l_i' < loss_{min}$}
         \State \Call{Branch-and-Bound}{$\l'$}
      \EndIf
      \State $\l' \gets$ \Call{Propagate-Loss}{$\l, i, 1 - \tilde{l}_i$}
      \If {$\sum_i l_i' < loss_{min}$}
         \State \Call{Branch-and-Bound}{$\l'$}
      \EndIf
   \EndIf
\EndProcedure
\Statex
\Function{Propagate-Loss}{$\l, i, lossValue$} 
   \State $\l' \gets \l$
   \State $l_i' \gets lossValue$   
   \State $\t' \gets $ targets prediction vector implied by $\l'$ 
   \State Let $\Phi =$ convex hull of $\{ \boldsymbol{x_k} \; | \; t_k'=t_i' \}$ 
   \If {$\exists \boldsymbol{x_j} \in \Phi$ s.t. $t_j' = -t_j$}
      \State $l_i' \gets +\infty$ \Comment{conflict -- infeasible}
   \Else
   \For{p:=1 \text{ {\bf to} } N}
      \If{$\boldsymbol{x_p} \in \Phi$ AND $l_p$ unassigned}
         \State $t_p' \gets t_i'$ \Comment{propagate assign}
         \State $l_p' \gets \mathbb{I} [t_p' \not= t_p]$
      \EndIf
   \EndFor
   \EndIf  
   \State \Return $\l'$ \Comment{implied loss vector assignments}
\EndFunction
\Statex
\EndFunction
\end{algorithmic}}
\vspace{-4mm}
\end{figure}
%=================================================

In Figure~\ref{alg:BnB.Final} we provide the full BnB algorithm that
searches for the optimal $\l^*$ from~\eqref{eq:bb_obj}.  The key idea
of the algorithm is to build a search tree of assignments while
tracking the minimum loss solution $\w^*$ and value $loss_{min}$ to
prune provably suboptimal branches of the search.  To make this
algorithm efficient, we propose the following heuristic improvements:

\noindent \emph{Initial Bound Approximation:} In line 2, we run a fast
SVM solver on the full data to obtain an initial best solution $\w^*$
and $loss_{min}$.  Clearly this should prune a large portion of the
search space and guarantees that \BB\ will do at least as well as
the SVM.

\noindent \emph{Decision Ordering:} It is well-known that the ordering
of decision and value assignments in \BB\ can drastically affect
search efficiency.  Fortunately, having an approximated decision
hyperplane from the initial SVM solution helps to determine the
assignment and value ordering. Under the assumption that the optimal
decision hyperplane is somewhat near the approximated hyperplane, the
loss values of points that lie closer to the approximated hyperplane
are more likely to be changed than those that are far away.  Hence in
line 14, the points $\x_i$ farthest from the initial approximated
decision hyperplane $\tilde{\w}$ are assigned first, and their inital
value is assigned to be consistent with the assignment $\tilde{l}_i$
of the initial SVM solution.

\noindent \emph{Loss Propagation:} In any partial assignment to $\l$,
points \emph{within} the convex hull defined by the set of $\x_i$
assigned true in $\l$ are implied to be true and similarly for the
false case.  Hence at every branch of \BB\ we utilize the technique of
\emph{forward checking} by calling \textsc{Propagate-Loss} (lines 15
abd 19) to augment $\l$ with any implied assigments that prune future
search steps or to detect whether an assignment conflict has been
found leading to infeasibility.  We also use the sum of forced and
assigned components of the loss vector as a lower bound for purposes
of pruning suboptimal branches (lines 16 and 19).

To detect whether a point $\p \in \R^D$ is an interior point
of the convex hull created by other points $\p_1,
\p_2, \dots, \p_k \in \R^D$, we approach the problem
using the Barycentric coordinate system.  Then point $\p$ is
an interior point of the convex hull created by points $\p_1,
\p_2, \dots, \p_k$ if the homogeneous barycentric
coordinates of $\p$ with respect to $\p_1,
\p_2, \dots, \p_k$ are all non-negative. This is
equivalent to checking whether the following linear constraints over
$\mathbf{u}$ are feasible (e.g., by calling an LP solver):
$$
u_i \geq 0 \text{ for } i = 1,2, \dots, k \land \sum_{i=1}^k u_i = 1 \land \sum_{i=1}^k u_i \p_i = \p. 
$$ If a feasible $\mathbf{u}$ exists, then point $\p$
lies in the convex hull of points $\p_1, \p_2, \dots,
\p_k$, otherwise it does not.

