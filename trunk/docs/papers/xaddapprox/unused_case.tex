%Now we define the basic operations that can be performed on symbolic functions in case representationIn the context of SDP, we need to perform operations on symbolic functions represented in case form. The operations we need are: scalar multiplication, addition, subtraction and multiplication of symbolic functions, case maximization, boolean restriction and continuous integration. This operations were a major contribution of \cite{sanner_uai11} and we refer to that for the detailed definitions. 

%%%%%%%%%%%%%%%%%%%%%%%%%%%%%%%%%%%%%%%%%%%%%%%%%%%%%%%%%%%%%%%%%%%%%%%%
%\incmargin{.5em}
%\linesnumbered
%\begin{algorithm}[th!]
%\vspace{-.5mm}
%\dontprintsemicolon
%\SetKwFunction{regress}{Regress}
%\SetKwFunction{approximate}{Approximate}
%\Begin{
%   $V^0:=0, h:=0$\;
%   \While{$h < H$}{
%       $h:=h+1$\;
%       \ForEach {$a(\vec{y}) \in A$}{
%              $Q_a^{h}(\vec{y})\,:=\,$\regress{$V^{h-1},a,\vec{y}$}\;
%              $Q_a^{h} := \max_{\vec{y}} \, Q_a^{h}(\vec{y})$ $\,$ \emph{// Continuous $\max$}\;
%              $V^{h} := \casemax_a \, Q_a^{h}$ $\,$ \emph{// $\casemax$ all $Q_a$}\;
%              $\pi^{*,h} := \argmax_{(a,\vec{y})} \, Q_a^{h}(\vec{y})$\;
%       }
%       $V^h = $\approximate{$V^{h}$}\; \label{approxline}
%       \If{$V^h = V^{h-1}$}
%           {break $\,$ \emph{// Terminate if early convergence}\;}
%   }
%     \Return{$(V^h,\pi^{*,h})$} \;
%}
%\caption{\footnotesize \texttt{VI}(Hybrid-MDP, $H$) $\longrightarrow$ $(V^h,\pi^{*,h})$ \label{alg:vi}}
%\vspace{-1mm}
%\end{algorithm}
%\decmargin{.5em}
%%%%%%%%%%%%%%%%%%%%%%%%%%%%%%%%%%%%%%%%%%%%%%%%%%%%%%%%%%%%%%%%%

\begin{comment}
Next we define the basic operations that can be applied on symbolic functions:

\emph{Unary operations} such as scalar multiplication $c\cdot f$ (for $c \in \mathbb{R}$) is simply applied to each $f_i$ ($1 \leq i \leq k$). Intuitively, to perform a \emph{binary
  operation} on two case statements, we simply take the cross-product
of the logical partitions of each case statement and perform the
corresponding operation on the resulting paired partitions. For instance, we define perform sum($\oplus$), subtraction($\ominus$) and product($\otimes$) by,
respectively, adding, subtracting or multiplying partition values to obtain the result. 

Next we define \emph{symbolic case maximization}:
{\footnotesize
\begin{center}
\begin{tabular}{r c c c l}
&
\hspace{0mm} $\casemax \Bigg(
  \begin{cases}
    \phi_1: \hspace{-2mm} & \hspace{-1mm} f_1 \\ 
    \phi_2: \hspace{-2mm} & \hspace{-1mm} f_2 \\ 
  \end{cases}$
$,$
&
\hspace{0mm}
  $\begin{cases}
    \psi_1: \hspace{-2mm} & \hspace{-1mm} g_1 \\ 
    \psi_2: \hspace{-2mm} & \hspace{-1mm} g_2 \\ 
  \end{cases} \Bigg)$
&
\hspace{0mm} 
$ = $
&
\hspace{0mm}
  $\begin{cases}
  \phi_1 \wedge \psi_1 \wedge f_1 > g_1    : & \hspace{-2mm} f_1 \\ 
  \phi_1 \wedge \psi_1 \wedge f_1 \leq g_1 : & \hspace{-2mm} g_1 \\ 
  \phi_1 \wedge \psi_2 \wedge f_1 > g_2    : & \hspace{-2mm}f_1 \\ 
  \phi_1 \wedge \psi_2 \wedge f_1 \leq g_2 : & \hspace{-2mm} g_2 \\ 
  \hspace{1cm}\vdots \hspace{8mm}: & \hspace{-2mm} \vdots
  \end{cases}$
\end{tabular}
\end{center}
} If all $f_i$ and $g_i$ are linear,
the $\casemax$ result is clearly still linear and representable in the case format previously described (i.e., linear inequalities in decisions). As linear inequalities can be inconsistent (infeasible) some of the generated partitions will be removed from the result. 

 The operation of \emph{boolean restriction} required for regressing boolean variables is obvious and an example is omitted: in this operation
$f|_{b=v}$, anywhere a boolean variable $b$ occurs in $f$, we assign
it the value $v \in \{ 0,1 \}$, and resolve the following logic implications, reducing formulae or removing partitions.  The operation of \emph{continuous integration} $\int Q(x'_j) \otimes P(x'_j|\cdots) dx'_j$ is required for regressing continuous variables. As previously defined, $P(x_j'|\cdots)$ will always be of the form $\delta[x_j' - h(\vec{z})]$ where
$h(\vec{z})$ is a case statement and $\vec{z}$ does not contain
$x'_j$.  Rules of integration then tell us that $\int f(x'_j) \otimes
\delta[x_j' - h(\vec{z})] dx'_j = f(x'_j) \{ x'_j / h(\vec{z}) \}$
where the latter operation indicates that any occurrence of $x'_j$ in
$f(x'_j)$ is \emph{symbolically substituted} with the case statement
$h(\vec{z})$. The appearance of case statement within formulae is solved expanding the cross-product of the logical partitions. A more detailed specification of this operation is available in ~\cite{sanner_uai11}.  The important insight is that this $\int$ operation yields a result that is a
case statement in the form previously outlined.

The only operation for SDP that has not yet been defined is the continuous action maximization.
The continuos maximization over action variables can be wrtitten as $g(\vec{b},\vec{x}) := \max_{\vec{y}} \, f(\vec{b},\vec{x},\vec{y})$. Note that the result of this max, $g(\vec{b},\vec{x})$ is a function of continuous variables, hence requiring \emph{symbolic} constrained optimization.

The definition of this maximization operation was the major contribution of~\cite{zamani12}, so refer to this work for a detailed explanation. We outline the basic steps for continuous maximization.

First, due to the commutativity of $\max$, any multivariate $\max_{\vec{y}}$ can be rewritten as a sequence of univariate $\max$ operations $\max_{y_1} \cdots \max_{y_{|\vec{y}|}}$; hence it
suffices to provide just the \emph{univariate} $\max_y$ solution:
$g(\vec{v}) := \max_{y} \, f(\vec{v},y)$, where $\vec{v}$ denotes a vector of boolean and continuous variables, in our case containing $\vec{b}$, $\vec{x}$ and other action parameters.

Second, because of the mutual disjointness of partitions, and the commutativity of $\casemax_i$ and $\max_y$, we can write $\max_y f(\vec{v},y) $ via:
{\footnotesize
\begin{align}
\max_y f(\vec{v},y) & = 
\max_y \casemax_i \, \phi_i(\vec{v},y) f_i(\vec{v},y) \nonumber \\
& = \casemax_i \, \fbox{$\max_y \phi_i(\vec{v},y) f_i(\vec{v},y)$} \label{eq:casemax_max}
\end{align}}

 Thus to complete this operation we need only
show how to symbolically compute the maximum in a single partition 
$\max_y \phi_i(\vec{v},y) f_i(\vec{v},y)$.

%In $\phi_i$, we observe that each conjoined constraint serves one of
%three purposes: (i) \emph{upper bound on $y$}: it can be written
%as $y < l(\vec{v})$; (ii) \emph{lower bound on $y$}: it can be written as $y >
%u(\vec{v})$ or (iii) \emph{independent of $y$}: the constraints do not contain $y$
%and can be safely factored outside of the $\max_y$, as $f_i(\vec{v},y) =  g_i(\vec{v})$.  
%Because there are multiple symbolic upper and lower
%bounds on $y$, we apply the $\casemax$ ($\casemin$) operator to determine the highest lower bound (lowest upper bound):

%The substitution operator $\{ y / f \}$ replaces $y$ with case statement $f$, 
%defined in~\cite{sanner_uai11}.

\end{comment}






